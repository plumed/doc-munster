In this page you can learn how to \hyperlink{_installation_ConfiguringPlumed}{configure}, \hyperlink{_installation_CompilingPlumed}{compile}, and \hyperlink{_installation_InstallingPlumed}{install} P\+L\+U\+M\+E\+D. For those of you who are impatient, the following might do the job\+: \begin{DoxyVerb}> ./configure --prefix=/usr/local
> make -j 4
> make doc # this is optional and requires doxygen installed
> make install # this is also optional - plumed can be used from the compilation directory
\end{DoxyVerb}


Once the above is completed you can use P\+L\+U\+M\+E\+D to analyze existing trajectories or you can play with the Lennard-\/\+Jones code that is included. However, because P\+L\+U\+M\+E\+D is mostly used to bias on the fly simulations performed with serious molecular dynamics packages, you can find instructions about how to \hyperlink{_installation_Patching}{patch } your favorite M\+D code so that it can be combined with P\+L\+U\+M\+E\+D below. Again, if you are impatient, something like this will do the job\+: \begin{DoxyVerb}> source /plumed/root/dir/sourceme.sh
> cd /md/root/dir
> plumed patch -p
\end{DoxyVerb}
 Then compile your M\+D code. For some M\+D codes these instructions are insufficient. It is thus recommended that you read the instructions at the end of this page. Notice that M\+D codes could in principle be \char`\"{}\+P\+L\+U\+M\+E\+D ready\char`\"{} in their official distribution. If your favorite M\+D code is available \char`\"{}\+P\+L\+U\+M\+E\+D ready\char`\"{} you will have to compile P\+L\+U\+M\+E\+D first as a library then check the M\+D codes' manual to discover how to link it.\hypertarget{_installation_ConfiguringPlumed}{}\section{Configuring P\+L\+U\+M\+E\+D}\label{_installation_ConfiguringPlumed}
The \char`\"{}./configure\char`\"{} command just generates a Makefile.\+conf file and a sourceme.\+sh file. In P\+L\+U\+M\+E\+D 2.\+0 these files were pre-\/prepared and stored in the directory configurations/. The new ones generated by ./configure should be compatible with the old ones. In other words, if you have difficulties with the new procedure, you can use one of these old configuration files. However, it should be easy to enforce a similar setup on autoconf by passing the proper arguments on the command line. We strongly encourage you to configure plumed in this way. If you have problems on your architecture, please report them to the mailing list.

Useful command line options for ./configure can be found by typing \begin{DoxyVerb}> ./configure --help
\end{DoxyVerb}
 Notice that some functionalities of P\+L\+U\+M\+E\+D depend on external libraries which are looked for by configure. You can typically avoid looking for a library using the \char`\"{}disable\char`\"{} syntax, e.\+g. \begin{DoxyVerb}> ./configure --disable-mpi --disable-matheval
\end{DoxyVerb}


Notice that when mpi search is enabled (by default) compilers such as \char`\"{}mpic++\char`\"{} and \char`\"{}mpicxx\char`\"{} are searched for first. On the other hand, if mpi search is disabled (\char`\"{}./configure -\/-\/disable-\/mpi\char`\"{}) non-\/mpi compilers are searched for. Notice that only a few of the possible compiler name are searched. Thus, compilers such as \char`\"{}g++-\/mp-\/4.\+8\char`\"{} should be explicitly requested with the C\+X\+X option.

You can better control which compiler is used by setting the variables C\+X\+X and C\+C. E.\+g., to use Intel compilers use the following command\+: \begin{DoxyVerb}> ./configure CXX=icpc CC=icc
\end{DoxyVerb}
 Notice that we are using icpc in this example, which is not an mpi compiler as a result mpi will not be enabled.

To tune the compilation options you can use the C\+X\+X\+F\+L\+A\+G\+S variable\+: \begin{DoxyVerb}> ./configure CXXFLAGS=-O3
\end{DoxyVerb}


If you are implementing new functionality and want to build with debug flags in place so as to do some checking you can use \begin{DoxyVerb}> ./configure --enable-debug
\end{DoxyVerb}
 This will perform some extra check during execution (possibly slowing down P\+L\+U\+M\+E\+D) and write full symbol tables in the executable (making the final executable much larger).

The main goal of the automatic configure is to find the libraries. When they are stored in unconventional places it is thus sensible to tell autoconf where to look! To do this there are some environment variable that can be used to instruct the linker which directories it should search for libraries inside. These variables are compiler dependent, but could have been set by the system administrator so that libraries are found without any extra flag. Our suggested procedure is to first try to configure without any additional flags and to then check the log so as to see whether or not the libraries were properly detected.

If a library is not found during configuration, you can try to use options to modify the seatch path. For example if your matheval libraries is in /opt/local (this is where Mac\+Ports put it) use \begin{DoxyVerb}> ./configure LDFLAGS=-L/opt/local/lib CPPFLAGS=-I/opt/local/include
\end{DoxyVerb}
 Notice that P\+L\+U\+M\+E\+D will first try to link a routine from say matheval without any additional flag, and then in case of failure will retry adding \char`\"{}-\/lmatheval\char`\"{} to the L\+I\+B\+S options. This allows you to use libraries with custom names. So, if your matheval library is called /opt/local/lib/libmymatheval.so you can link it with \begin{DoxyVerb}> ./configure LDFLAGS=-L/opt/local/lib CPPFLAGS=-I/opt/local/include LIBS=-lmymatheval
\end{DoxyVerb}
 In this example, if the linker finds the libmymatheval.\+so library it will be happy. If not it will try adding \char`\"{}-\/lmatheval\char`\"{}. If also this does not work, the matheval library will be disabled and some features will not be available. This rule is true for all the libraries, so that you will always be able to link a specific version of a library by specifying it using the L\+I\+B\+S variable.

\begin{DoxyWarning}{Warning}
On Linux you might have problems using the L\+D\+F\+L\+A\+G\+S option. In particular, if you have problems in linking the file 'src/lib/plumed-\/shared', try to set correctly the runtime path by using \begin{DoxyVerb}> ./configure LDFLAGS="-L/opt/local/lib -Wl,-rpath,/opt/local/lib" \
  CPPFLAGS=-I/opt/local/include LIBS=-lmymatheval
\end{DoxyVerb}
 Notice that although the file 'src/lib/plumed-\/shared' is not necessary, being able to produce it means that it will be possible to link P\+L\+U\+M\+E\+D dynamically with M\+D codes later.
\end{DoxyWarning}
P\+L\+U\+M\+E\+D needs blas and lapack. These are treated slighty different from other libraries. The search is done in the usual way (i.\+e., first look for them without any link flag, then add \char`\"{}-\/lblas\char`\"{} and \char`\"{}-\/llapack\char`\"{}, resepctively). As such if you want to use a specific version of blas or lapack you can make them available to configure by using \begin{DoxyVerb}> ./configure LDFLAGS=-L/path/to/blas/lib LIBS=-lnameoflib
\end{DoxyVerb}
 If the functions of these libraries are not found, the compiler looks for a version with a final underscore added. Finally, since blas and lapack are compulsory in P\+L\+U\+M\+E\+D, you can use a internal version of these libraries that comes as part of P\+L\+U\+M\+E\+D. If all else fails the internal version of B\+L\+A\+S and L\+A\+P\+A\+C\+K are the ones that will be used by P\+L\+U\+M\+E\+D. If you wish to disable any search for external libraries (e.\+g. because the system libraries have problems) this can be done with \begin{DoxyVerb}> ./configure --disable-external-lapack
\end{DoxyVerb}


As a final resort, you can also edit the resulting Makefile.\+conf file. Notable variables in this file include\+:
\begin{DoxyItemize}
\item D\+Y\+N\+A\+M\+I\+C\+\_\+\+L\+I\+B \+: these are the libraries needed to compile the P\+L\+U\+M\+E\+D library (e.\+g. -\/\+L/path/to/matheval -\/lmatheval etc). Notice that for the P\+L\+U\+M\+E\+D shared library to be compiled properly these should be dynamic libraries. Also notice that P\+L\+U\+M\+E\+D preferentially requires B\+L\+A\+S and L\+A\+P\+A\+C\+K library; see \hyperlink{_installation_BlasAndLapack}{B\+L\+A\+S and L\+A\+P\+A\+C\+K} for further info. Notice that the variables that you supply with {\ttfamily configure L\+I\+B\+S=something} will end up in this variable. This is a bit misleading but is required to keep the configuration files compatible with P\+L\+U\+M\+E\+D 2.\+0.
\item L\+I\+B\+S \+: these are the libraries needed when patching an M\+D code; typically only \char`\"{}-\/ldl\char`\"{} (needed to have functions for dynamic loading).
\item C\+P\+P\+F\+L\+A\+G\+S \+: add here definition needed to enable specific optional functions; e.\+g. use -\/\+D\+\_\+\+\_\+\+P\+L\+U\+M\+E\+D\+\_\+\+H\+A\+S\+\_\+\+M\+A\+T\+H\+E\+V\+A\+L to enable the matheval library
\item S\+O\+E\+X\+T \+: this gives the extension for shared libraries in your system, typically \char`\"{}so\char`\"{} on unix, \char`\"{}dylib\char`\"{} on mac; If your system does not support dynamic libraries or, for some other reason, you would like only static executables you can just set this variable to a blank (\char`\"{}\+S\+O\+E\+X\+T=\char`\"{}).
\end{DoxyItemize}\hypertarget{_installation_BlasAndLapack}{}\subsection{B\+L\+A\+S and L\+A\+P\+A\+C\+K}\label{_installation_BlasAndLapack}
We tried to keep P\+L\+U\+M\+E\+D as independent as possible from external libraries and as such those features that require external libraries (e.\+g. Almost and Matheval) are optional. However, to have a properly working version of plumed P\+L\+U\+M\+E\+D you need B\+L\+A\+S and L\+A\+P\+A\+C\+K libraries. We would strongly recommend you download these libraries and install them separately so as to have the most efficient possible implementations of the functions contained within them. However, if you cannot install blas and lapack, you can use the internal ones. Since version 2.\+1, P\+L\+U\+M\+E\+D uses a configure script to detect libraries. In case system L\+A\+P\+A\+C\+K or B\+L\+A\+S are not found on your system, P\+L\+U\+M\+E\+D will use the internal replacement.

We have had a number of emails (and have struggled ourselves) with ensuring that P\+L\+U\+M\+E\+D can link B\+L\+A\+S and L\+A\+P\+A\+C\+K. The following describes some of the pitfalls that you can fall into and a set of sensible steps by which you can check whether or not you have set up the configuration correctly.

Notice first of all that the D\+Y\+N\+A\+M\+I\+C\+\_\+\+L\+I\+B variable in the Makefile.\+conf should contain the flag necessary to load the B\+L\+A\+S and L\+A\+P\+A\+C\+K libraries. Typically this will be -\/llapack -\/lblas, in some case followed by -\/lgfortran. Full path specification with -\/\+L may be necessary and on some machines the blas and lapack libraries may not be called -\/llapack and -\/lblas. Everything will depend on your system configuration.

Some simple to fix further problems include\+:
\begin{DoxyItemize}
\item If the linker complains and suggests recompiling lapack with -\/f\+P\+I\+C, it means that you have static lapack libraries. Either install dynamic lapack libraries or switch to static compilation of P\+L\+U\+M\+E\+D by unsetting the S\+O\+E\+X\+T variable in the configuration file.
\item If the linker complains about other missing functions (typically starting with \char`\"{}for\+\_\+\char`\"{} prefix) then you should also link some Fortran libraries. P\+L\+U\+M\+E\+D is written in C++ and often C++ linkers do not include Fortran libraries by default. These libraries are required for lapack and blas to work. Please check the documentation of your compiler.
\item If the linker complains that dsyevr\+\_\+ cannot be found, try adding -\/\+D\+F77\+\_\+\+N\+O\+\_\+\+U\+N\+D\+E\+R\+S\+C\+O\+R\+E to C\+P\+P\+F\+L\+A\+G\+S Notice that \char`\"{}./configure\char`\"{} should automatically try this solution.
\end{DoxyItemize}\hypertarget{_installation_CompilingPlumed}{}\section{Compiling P\+L\+U\+M\+E\+D}\label{_installation_CompilingPlumed}
Once configured, P\+L\+U\+M\+E\+D can be compiled using the following command\+: \begin{DoxyVerb}> make -j 4
\end{DoxyVerb}
 This will compile the entire code and produce a number of files in the 'src/lib' directory, including the executable 'src/lib/plumed'. When shared libraries are enabled, a shared libraries called 'src/lib/lib\+Kernel.\+so' should also have been compiled. Notice that the extension could be '.dylib' on a Mac.

The file 'sourceme.\+sh' that has been created by the configure script in the main P\+L\+U\+M\+E\+D directory can be \char`\"{}sourced\char`\"{} (presently only working for bash shell) if you want to use P\+L\+U\+M\+E\+D {\itshape without installing it} (i.\+e. from the compilation directory). It is a good idea to source it\+: \begin{DoxyVerb}> source sourceme.sh
\end{DoxyVerb}


If compilation is successful, a \char`\"{}plumed\char`\"{} executable should be in your path. Try to type \begin{DoxyVerb}> plumed -h
\end{DoxyVerb}


\begin{DoxyWarning}{Warning}
If you are cross compiling, the plumed executable will not work. As a consequence, you won't be able to run regtests or compile the manual.
\end{DoxyWarning}
You can also check if P\+L\+U\+M\+E\+D is correctly compiled by performing our regression tests. Be warned that some of them fail because of the different numerical accuracy on different machines. \begin{DoxyVerb}> cd regtest
> make
\end{DoxyVerb}
 Notice that regtests are performed using the \char`\"{}plumed\char`\"{} executable that is currenty in the path. You can check the exact version they will use by using the command \begin{DoxyVerb}> which plumed
\end{DoxyVerb}
 This means that if you do not source \char`\"{}sourceme.\+sh\char`\"{}, the tests will fails. This does not mean that plumed is not working it just means that you haven't told them shell where to find plumed!

Notice that the compiled executable, which now sits in 'src/lib/plumed', relies on other resource files present in the compilation directory. This directory should thus stay in the correct place. One should thus not rename or delete it. In fact the path to the P\+L\+U\+M\+E\+D root directory is hardcoded in the plumed executable as can be verified using \begin{DoxyVerb}> plumed info --root
\end{DoxyVerb}
 In case you try to use the plumed executable without the compilation directory in place (e.\+g. you move away the src/lib/plumed static executable and delete or rename the compilation directory) P\+L\+U\+M\+E\+D will not work correctly and will give you an error message \begin{DoxyVerb}> plumed help
ERROR: I cannot find /xxx/yyy/patches directory
\end{DoxyVerb}
 You can force plumed to run anyway by using the option --standalone-\/executable\+: \begin{DoxyVerb}> plumed --standalone-executable help
\end{DoxyVerb}
 Many features will not be available if you run in this way. However, this is currently the only way to use the P\+L\+U\+M\+E\+D static executable on Windows.\hypertarget{_installation_InstallingPlumed}{}\section{Installing P\+L\+U\+M\+E\+D}\label{_installation_InstallingPlumed}
It might be convenient to install P\+L\+U\+M\+E\+D in a predefined location. This will allow you to remove the original compilation directory, or to recompile a different P\+L\+U\+M\+E\+D version in the same place. Notice that installation {\itshape is optional}. Even from the compilation directory, if the environment is properly set (see sourceme.\+sh file) P\+L\+U\+M\+E\+D should work.

To install P\+L\+U\+M\+E\+D one should first decide the location. Just set the environment variable P\+L\+U\+M\+E\+D\+\_\+\+P\+R\+E\+F\+I\+X, then type \char`\"{}make install\char`\"{} \begin{DoxyVerb}> export PLUMED_PREFIX=$HOME/opt
> make install
\end{DoxyVerb}
 If P\+L\+U\+M\+E\+D\+\_\+\+P\+R\+E\+F\+I\+X is not set, it will be assumed to be the one set when you configured with autoconf. So if you configured using \begin{DoxyVerb}> ./configure --prefix=$HOME/opt
> make
> make install
\end{DoxyVerb}
 Then the P\+L\+U\+M\+E\+D\+\_\+\+P\+R\+E\+F\+I\+X will be set equal to \$\+H\+O\+M\+E/opt. If the P\+L\+U\+M\+E\+D\+\_\+\+P\+R\+E\+F\+I\+X is not set, it defaults to /usr/local. The install command should be executed with root permissions (e.\+g. \char`\"{}sudo make install\char`\"{}) if you want to install P\+L\+U\+M\+E\+D on a system directory. Notice that upon installation P\+L\+U\+M\+E\+D currently needs to relink a library. If root user does not have access to compilers, \char`\"{}sudo -\/\+E make install\char`\"{} might solve the issue. An almost full copy of the compilation directory will be installed into \$\+P\+L\+U\+M\+E\+D\+\_\+\+P\+R\+E\+F\+I\+X/lib/plumed/ directory. A link to the proper P\+L\+U\+M\+E\+D executable will be set up in \$\+P\+L\+U\+M\+E\+D\+\_\+\+P\+R\+E\+F\+I\+X/bin, P\+L\+U\+M\+E\+D include files will be copied to \$\+P\+L\+U\+M\+E\+D\+\_\+\+P\+R\+E\+F\+I\+X/include/plumed and P\+L\+U\+M\+E\+D libraries will be linked to \$\+P\+L\+U\+M\+E\+D\+\_\+\+P\+R\+E\+F\+I\+X/lib.

One should then set the environment properly. We suggest to do it using the module framework (\href{http://modules.sourceforge.net}{\tt http\+://modules.\+sourceforge.\+net}). An ad hoc generated module file for P\+L\+U\+M\+E\+D can be found in \$\+P\+L\+U\+M\+E\+D\+\_\+\+P\+R\+E\+F\+I\+X/lib/plumed/src/lib/modulefile Just edit it as you wish and put it in your modulefile directory. This will also allow you to install multiple P\+L\+U\+M\+E\+D versions on your machine and to switch amongst them. If you do not want to use modules, you can still have a look at the modulefile we did so as to know which environment variables should be set for P\+L\+U\+M\+E\+D to work correctly.

If the environment is properly configured one should be able to do the following things\+:
\begin{DoxyItemize}
\item use the \char`\"{}plumed\char`\"{} executable from the command line. This is also possible before installing.
\item link against the P\+L\+U\+M\+E\+D library using the \char`\"{}-\/lplumed\char`\"{} flag for the linker. This allows one to use P\+L\+U\+M\+E\+D library in general purpose programs
\item use the P\+L\+U\+M\+E\+D internal functionalities (C++ classes) including header files such as \char`\"{}\#include $<$plumed/tools/\+Vector.\+h$>$\char`\"{}. This is useful as it may be expedient to exploit the P\+L\+U\+M\+E\+D library in general purpose programs
\end{DoxyItemize}

As a final note, if you want to install several P\+L\+U\+M\+E\+D versions without using modules then you can define the environment variable P\+L\+U\+M\+E\+D\+\_\+\+L\+I\+B\+S\+U\+F\+F\+I\+X using\+: \begin{DoxyVerb}> export PLUMED_PREFIX=$HOME/opt
> export PLUMED_LIBSUFFIX=v2.0
> make install
\end{DoxyVerb}
 This will install a plumed executable named \char`\"{}plumed-\/v2.\+0\char`\"{}. All the other files will be renamed similarly, e.\+g. the P\+L\+U\+M\+E\+D library will be loaded with \char`\"{}-\/lplumed-\/v2.\+0\char`\"{} and the P\+L\+U\+M\+E\+D header files will be included with \char`\"{}\#include $<$plumed-\/v2.\+0/tools/\+Vector.\+h$>$\char`\"{}. This trick is useful if you do not want to set up modules, but we believe that using modules as described above is more flexible.\hypertarget{_installation_Patching}{}\section{Patching your M\+D code}\label{_installation_Patching}
In case your M\+D code is not supporting P\+L\+U\+M\+E\+D already, you should modify it. We provide scripts to adjust some of the most popular M\+D codes so as to provide P\+L\+U\+M\+E\+D support. At the present times we support patching the following list of codes\+:


\begin{DoxyItemize}
\item amber14
\item gromacs-\/4-\/5-\/5
\item gromacs-\/4-\/6-\/7
\item gromacs-\/5-\/0
\item lammps-\/6\+Apr13
\item namd-\/2-\/8
\item namd-\/2-\/9
\item qespresso-\/5-\/0-\/2
\end{DoxyItemize}

In the section \hyperlink{CodeSpecificNotes}{Code specific notes} you can find information specific for each M\+D code.

To patch your M\+D code, you should have already installed P\+L\+U\+M\+E\+D properly. This is necessary as you need to have the command \char`\"{}plumed\char`\"{} in your execution path. As described above this executible will be in your paths if plumed was installed or if you have run sourceme.\+sh

Once you have a compiled and working version of plumed, follow these steps to add it to an M\+D code
\begin{DoxyItemize}
\item Configure and compile your M\+D enginge (look for the instructions in its documentation).
\item Test if the M\+D code is working properly.
\item Go to the root directory for the source code of the M\+D engine.
\item Patch with P\+L\+U\+M\+E\+D using\+: \begin{DoxyVerb}> plumed patch -p
\end{DoxyVerb}
 The script will interactively ask which M\+D engine you are patching.
\item Once you have patched recompile the M\+D code (if dependencies are set up properly in the M\+D engine, only modified files will be recompiled)
\end{DoxyItemize}

There are different options available when patching. You can check all of them using \begin{DoxyVerb}> plumed patch --help
\end{DoxyVerb}
 Particularly interesting options include\+:
\begin{DoxyItemize}
\item --static (default) just link P\+L\+U\+M\+E\+D as a collection of object files.
\item --shared allows you to link P\+L\+U\+M\+E\+D as a shared library. As a result when P\+L\+U\+M\+E\+D is updated, there will be no need to recompile the M\+D code.
\item --runtime allows you to choose the location of the P\+L\+U\+M\+E\+D library at runtime by setting the variable P\+L\+U\+M\+E\+D\+\_\+\+K\+E\+R\+N\+E\+L.
\end{DoxyItemize}

Notice that it is not currently possible to link P\+L\+U\+M\+E\+D as a static library (something like 'libplumed.\+a'). The reason for this is that P\+L\+U\+M\+E\+D heavily relies on C++ static constructors that do not behave well in static libraries. For this reason, to produce a static executable with an M\+D code + P\+L\+U\+M\+E\+D we link P\+L\+U\+M\+E\+D as a collection of object files.

A note for cross compiling\+: if you are compiling an executable from a different machine, then then \char`\"{}plumed\char`\"{} executable will not be available in the compilation environment. You should thus use the following command \begin{DoxyVerb}> plumed-patch
\end{DoxyVerb}
 as a replacement for \char`\"{}plumed patch\char`\"{}.

If your M\+D code is not supported, you may want to implement an interface for it. Refer to the \href{../../developer-doc/html/index.html}{\tt developer manual }.\hypertarget{_installation_installingalmost}{}\section{Installing P\+L\+U\+M\+E\+D with A\+L\+M\+O\+S\+T}\label{_installation_installingalmost}
In order to used some of the N\+M\+R based collective variables (\hyperlink{CS2BACKBONE}{C\+S2\+B\+A\+C\+K\+B\+O\+N\+E} and \hyperlink{CH3SHIFTS}{C\+H3\+S\+H\+I\+F\+T\+S}) P\+L\+U\+M\+E\+D needs to be linked with A\+L\+M\+O\+S\+T. To do this the free package A\+L\+M\+O\+S\+T v.\+2.\+1 M\+U\+S\+T be dowloaded via S\+V\+N (svn checkout svn\+://svn.code.\+sf.\+net/p/almost/code/ almost-\/code). A\+L\+M\+O\+S\+T 2.\+1 can be found in branches/almost-\/2.\+1/ and can be compiled\+:

\begin{DoxyWarning}{Warning}
A\+L\+M\+O\+S\+T needs S\+Q\+L\+I\+T\+E3 and G\+Z\+I\+P installed on your computer. 

A\+L\+M\+O\+S\+T cannot be installed in the same folder of the source code, use --prefix to install it in a different folder
\end{DoxyWarning}
\begin{DoxyVerb}> ./configure --prefix="wherever you want it" CXXFLAGS="-O3 -fPIC" CFLAGS="-O3 -fPIC" 
> make
> make install
\end{DoxyVerb}


Sometimes A\+L\+M\+O\+S\+T can give errors related to the automake tools. To fix them it is often enough to execute \begin{DoxyVerb}> autoreconf -fi
> automake
\end{DoxyVerb}
 and then repeat the configuration and compilation instructions.

P\+L\+U\+M\+E\+D will not use the R\+D\+Cs module of A\+L\+M\+O\+S\+T so you can ignore the warning about L\+A\+P\+A\+C\+K.

Once A\+L\+M\+O\+S\+T is installed, P\+L\+U\+M\+E\+D 2 can then be configured with A\+L\+M\+O\+S\+T enabled\+:

\begin{DoxyVerb}> ./configure --enable-almost CPPFLAGS="-I/ALMOST_INSTALL_PATH/include \
  -I/ALMOST_INSTALL_PATH/include/almost" LDFLAGS="-L/ALMOST_INSTALL_PATH/lib"\end{DoxyVerb}
 with A\+L\+M\+O\+S\+T\+\_\+\+I\+N\+S\+T\+A\+L\+L\+\_\+\+P\+A\+T\+H set to the full path to the A\+L\+M\+O\+S\+T installation folder. \hypertarget{CodeSpecificNotes}{}\section{Code specific notes}\label{CodeSpecificNotes}
Here you can find instructions that are specific for patching each of the supported M\+D codes.


\begin{DoxyItemize}
\item \hyperlink{amber14}{amber14}
\item \hyperlink{gromacs-4-5-5}{gromacs-\/4.\+5.\+5}
\item \hyperlink{gromacs-4-6-7}{gromacs-\/4.\+6.\+7}
\item \hyperlink{gromacs-5-0}{gromacs-\/5.\+0}
\item \hyperlink{lammps-6Apr13}{lammps-\/6\+Apr13}
\item \hyperlink{namd-2-8}{namd-\/2.\+8}
\item \hyperlink{namd-2-9}{namd-\/2.\+9}
\item \hyperlink{qespresso-5-0-2}{qespresso-\/5.\+0.\+2} 
\end{DoxyItemize}\hypertarget{amber14}{}\subsection{amber14}\label{amber14}
P\+L\+U\+M\+E\+D can be incorporated into amber (sander module) using the standard patching procedure. Patching must be done in the root directory of amber {\itshape before} compilation.

To enable P\+L\+U\+M\+E\+D in a sander simulation one should use add to the cntrl input namelist these two fields\+:

plumed=1 , plumedfile='plumed.\+dat'

The first is switching plumed on, the second is specifying the name of the plumed input file.

This patch is compatible with the M\+P\+I version of sander and support multisander. However, replica exchange is not supported. Multisander can thus only be used for multiple walkers metadynamics or for ensemble restraints.

For more information on amber you should visit \href{http://ambermd.org}{\tt http\+://ambermd.\+org} \hypertarget{gromacs-4-5-5}{}\subsection{gromacs-\/4.5.5}\label{gromacs-4-5-5}
P\+L\+U\+M\+E\+D can be incorporated into gromacs using the standard patching procedure. Patching must be done in the gromacs source directory {\itshape after} gromacs has been configured but {\itshape before} gromacs is compiled. Gromcas should be configured with ./configure (not cmake).

To enable P\+L\+U\+M\+E\+D in a gromacs simulation one should use mdrun with an extra -\/plumed flag. The flag can be used to specify the name of the P\+L\+U\+M\+E\+D input file, e.\+g.\+:

mdrun -\/plumed plumed.\+dat

For more information on gromacs you should visit \href{http://www.gromacs.org}{\tt http\+://www.\+gromacs.\+org} \hypertarget{gromacs-4-6-7}{}\subsection{gromacs-\/4.6.7}\label{gromacs-4-6-7}
P\+L\+U\+M\+E\+D can be incorporated into gromacs using the standard patching procedure. Patching must be done in the gromacs root directory {\itshape before} the cmake command is invoked.

To enable P\+L\+U\+M\+E\+D in a gromacs simulation one should use mdrun with an extra -\/plumed flag. The flag can be used to specify the name of the P\+L\+U\+M\+E\+D input file, e.\+g.\+:

mdrun -\/plumed plumed.\+dat

For more information on gromacs you should visit \href{http://www.gromacs.org}{\tt http\+://www.\+gromacs.\+org} \hypertarget{gromacs-5-0}{}\subsection{gromacs-\/5.0}\label{gromacs-5-0}
P\+L\+U\+M\+E\+D can be incorporated into gromacs using the standard patching procedure. Patching must be done in the gromacs root directory {\itshape before} the cmake command is invoked.

To enable P\+L\+U\+M\+E\+D in a gromacs simulation one should use mdrun with an extra -\/plumed flag. The flag can be used to specify the name of the P\+L\+U\+M\+E\+D input file, e.\+g.\+:

gmx mdrun -\/plumed plumed.\+dat

For more information on gromacs you should visit \href{http://www.gromacs.org}{\tt http\+://www.\+gromacs.\+org} \hypertarget{lammps-6Apr13}{}\subsection{lammps-\/6\+Apr13}\label{lammps-6Apr13}
P\+L\+U\+M\+E\+D can be incorporated into L\+A\+M\+M\+P\+S using a simple patching procedure. Patching must be done {\itshape before} L\+A\+M\+M\+P\+S is configured. After patching, one should enable P\+L\+U\+M\+E\+D using the command make yes-\/user-\/plumed In the same way, before reverting one should disable P\+L\+U\+M\+E\+D using the command make no-\/user-\/plumed

Also notice that command \char`\"{}fix plumed\char`\"{} should be used in lammps input file {\itshape after} the relevant input parameters have been set (e.\+g. after \char`\"{}timestep\char`\"{} command)

See also \href{http://lammps.sandia.gov/doc/Section_commands.html}{\tt http\+://lammps.\+sandia.\+gov/doc/\+Section\+\_\+commands.\+html} for further info on processing L\+A\+M\+M\+P\+S input, as well as this discussion on github\+: \href{http://github.com/plumed/plumed2/issues/67}{\tt http\+://github.\+com/plumed/plumed2/issues/67}.

For more information on L\+A\+M\+M\+P\+S you should visit \href{http://lammps.sandia.gov/}{\tt http\+://lammps.\+sandia.\+gov/} \hypertarget{namd-2-8}{}\subsection{namd-\/2.8}\label{namd-2-8}
\begin{DoxyRefDesc}{Bug}
\item[\hyperlink{bug__bug000004}{Bug}]N\+A\+M\+D does not currently take into account virial contributions from P\+L\+U\+M\+E\+D. Please use constant volume simulations only\end{DoxyRefDesc}


For more information on N\+A\+M\+D you should visit \href{http://www.ks.uiuc.edu/Research/namd/}{\tt http\+://www.\+ks.\+uiuc.\+edu/\+Research/namd/} \hypertarget{namd-2-9}{}\subsection{namd-\/2.9}\label{namd-2-9}
\begin{DoxyRefDesc}{Bug}
\item[\hyperlink{bug__bug000005}{Bug}]N\+A\+M\+D does not currently take into account virial contributions from P\+L\+U\+M\+E\+D. Please use constant volume simulations only\end{DoxyRefDesc}


For more information on N\+A\+M\+D you should visit \href{http://www.ks.uiuc.edu/Research/namd/}{\tt http\+://www.\+ks.\+uiuc.\+edu/\+Research/namd/} \hypertarget{qespresso-5-0-2}{}\subsection{qespresso-\/5.0.2}\label{qespresso-5-0-2}
For more information on Quantum Espresso you should visit \href{http://www.quantum-espresso.org}{\tt http\+://www.\+quantum-\/espresso.\+org} 